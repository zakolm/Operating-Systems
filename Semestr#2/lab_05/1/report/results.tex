% !TEX root = main.tex
\section{Структура FILE}

%\biglisting{../FILE.c}
		\lstinputlisting[
		language=C,                 % выбор языка для подсветки (здесь это С)
		basicstyle=\small\sffamily, % размер и начертание шрифта для подсветки кода
		numbers=left,               % где поставить нумерацию строк (слева\справа)
		numberstyle=\tiny,           % размер шрифта для номеров строк
		stepnumber=1,                   % размер шага между двумя номерами строк
		numbersep=5pt,                % как далеко отстоят номера строк от подсвечиваемого кода
		backgroundcolor=\color{white}, % цвет фона подсветки - используем \usepackage{color}
		showspaces=false,            % показывать или нет пробелы специальными отступами
		showstringspaces=false,      % показывать или нет пробелы в строках
		showtabs=false,             % показывать или нет табуляцию в строках
		frame=single,              % рисовать рамку вокруг кода
		tabsize=2,                 % размер табуляции по умолчанию равен 2 пробелам
		captionpos=t,              % позиция заголовка вверху [t] или внизу [b] 
		breaklines=true,           % автоматически переносить строки (да\нет)
		breakatwhitespace=false, % переносить строки только если есть пробел
		escapeinside={\%*}{*)},  % если нужно добавить комментарии в коде
		caption=Структура FILE для OS X.
		]{../FILE.c}

\section{Вывод}

Исходя из вышеприведенных рассуждений, можно сделать несколько выводов.

\begin{enumerate}
	\item  При буфферизованном вводе/выводе необходимо учитывать факт записи/чтения данных из буффера, т.к. неправильные действия с данными, записываемыми(или считываемыми) в(из) файл(а), могут привести к неправильной последовательности данных(пример 1) или даже к их потере(пример 3).
	\item При небуффиризованном вводе/выводе необходимо учитывать, что при одновременном открытии одного и того же файла создается дескриптор открытого файла(столько дескрипторов, сколько раз был открыт файл). Каждый дескриптор struct\_file имеет поле f\_pos, указывающий на позицию чтения или записи в \underline{логическом файле}.
 \end{enumerate}
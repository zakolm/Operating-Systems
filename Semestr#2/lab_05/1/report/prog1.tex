% !TEX root = main.tex
\section{CIO.c}

\paragraph{Текст программы}\hfill\\

\biglisting{../CIO.c}


\paragraph{Вывод}\hfill\\

\biglisting{../CIO_output.txt}

\paragraph{Анализ}\hfill\\
Анализ работы программы и результатов: в самом начале программы создается дескриптор файла “alphabet.txt” в таблице файловых дескрипторов процесса и запись в таблице открытых файлов. В записи этой таблицы хранится текущее смещение указателя в файле, которое используется во всех операциях чтения и записи в файл,
а также режим открытия файла (O\_RDONLY, O\_WRONLY или O\_RDWR). При выполнении операции чтения-записи система выполняет неявный сдвиг указателя. Далее создаются два объекта типа FILE которые ссылаются на созданный файловый дескриптор и при помощи функцией $setvbuf(…)$ мы задаем размер буфера каждого объекта равный 20 символам.
При первом вызове $fscanf(fs1,"\%c",\&c)$, буфер первого файлового дескриптора предварительно заполняется первыми 20 символами - abcdefghijklmnopqrst,
а при втором $fscanf(fs2,"\%c",\&c)$ буфер второго файлового дескриптора заполняется оставшейся частью – uvwxyz (т. к. обе структуры FILE ссылаются на одну и ту же запись в таблице открытых файлов, а при первом чтении указатель в файле уже сместился на 20 позиций). Далее в цикле происходит поочередная печать по одному символу, который уже берется не из файла, а из соответствующего буфера файлового дескриптора. Поэтому мы можем наблюдать такие результаты.
 
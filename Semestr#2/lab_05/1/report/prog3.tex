% !TEX root = main.tex
\section{FOpen.c}

\biglisting{../FOpen.c}

\paragraph{Вывод}\hfill\\

\biglisting{../FOpen_output.txt}


\paragraph{Анализ}\hfill\\

при вызове функции fopen() создаются 2 файловых дескриптора и 2 записи в таблице открытых файлов. Вспомним, что при вызове fprintf(...), запись производится в буфер.И только тогда, когда буфер будет заполнен полностью или если, будут вызваны функции fclose(...), fflush(...) данные будут записаны в файл. По этой причине, когда в программе вызвалась функция fclose(fs1), в файл записались все буквы английского алфавита, которые были в буфере файлового дескриптора fs1 (acegikmoqsuwy), тогда как после вызова функции fclose(fs2), содержимое файла удалилось (т.к. мы открыли файл для записи с режимом “w”), и записалась информация из буфера fs2.
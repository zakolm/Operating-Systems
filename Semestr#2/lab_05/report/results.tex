% !TEX root = main.tex
\section{Структура FILE}

%\biglisting{../FILE.c}
		\lstinputlisting[
		language=C,                 % выбор языка для подсветки (здесь это С)
		basicstyle=\small\sffamily, % размер и начертание шрифта для подсветки кода
		numbers=left,               % где поставить нумерацию строк (слева\справа)
		numberstyle=\tiny,           % размер шрифта для номеров строк
		stepnumber=1,                   % размер шага между двумя номерами строк
		numbersep=5pt,                % как далеко отстоят номера строк от подсвечиваемого кода
		backgroundcolor=\color{white}, % цвет фона подсветки - используем \usepackage{color}
		showspaces=false,            % показывать или нет пробелы специальными отступами
		showstringspaces=false,      % показывать или нет пробелы в строках
		showtabs=false,             % показывать или нет табуляцию в строках
		frame=single,              % рисовать рамку вокруг кода
		tabsize=2,                 % размер табуляции по умолчанию равен 2 пробелам
		captionpos=t,              % позиция заголовка вверху [t] или внизу [b] 
		breaklines=true,           % автоматически переносить строки (да\нет)
		breakatwhitespace=false, % переносить строки только если есть пробел
		escapeinside={\%*}{*)},  % если нужно добавить комментарии в коде
		caption=Структура FILE для OS X.
		]{../FILE.c}

\section{Вывод}

Исходя из вышеприведенных рассуждений, можно сделать несколько выводов.

\begin{enumerate}
	\item  Предпочтительней использовать функцию $fopen()$, т.к. $fopen()$ выполняет ввод-вывод с буферизацией, что может оказаться значительно быстрее, чем  с использованием $open()$, $FILE *$ дает возможность использовать $fscanf()$ и другие функции $stdio.h$.
	\item Следует помнить о буферизации и вовремя использовать $fclose()$ для записи в файл.
	\item С острожностью использовать $fflush()$, т.к. она оставляет поток открытым.
	\item Необходимо следить за режимом, с котором открывается поток. 
	\item Созданный новый дескриптор открытого файла  изначально не разделяется с любым другим процессом, но разделение может возникнуть через $fork()$.
	\item Функции $fscanf, fprintf, fopen, fclose$ являются обертками высшего уровня над системными вызовами $open, close, read, write$.
\end{enumerate}